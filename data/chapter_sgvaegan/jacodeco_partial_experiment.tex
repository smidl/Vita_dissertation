\begin{table}[h] 
 \center 
 \begin{tabular}{c c c c c } 
 \toprule 
  class & AUC - no $J_d(x)$ & AUC - with $J_d(x)$ & $\alpha_r$ & $\alpha_j$  \\ 
  \midrule
  0 & 0.70 $\pm $0.04 & 0.70 $\pm $0.05 & 1.00 & 0.00  \\ 
  1 & 0.82 $\pm $0.01 & 0.82 $\pm $0.02 & 1.09 & -0.03  \\ 
  2 & 0.71 $\pm $0.02 & 0.72 $\pm $0.02 & 0.93 & -0.01  \\ 
  3 & 0.64 $\pm $0.02 & 0.64 $\pm $0.02 & 0.94 & 0.01  \\ 
  4 & 0.72 $\pm $0.03 & 0.72 $\pm $0.03 & 1.00 & 0.00  \\ 
  5 & 0.67 $\pm $0.01 & 0.66 $\pm $0.01 & 0.98 & 0.00  \\ 
  6 & 0.68 $\pm $0.02 & 0.68 $\pm $0.02 & 0.60 & 0.00  \\ 
  7 & 0.73 $\pm $0.05 & 0.73 $\pm $0.05 & 1.00 & 0.00  \\ 
  8 & 0.69 $\pm $0.04 & 0.71 $\pm $0.02 & 0.98 & -0.04  \\ 
  9 & 0.63 $\pm $0.05 & 0.63 $\pm $0.04 & 0.80 & 0.00  \\ 
  \bottomrule
 \end{tabular}
 \caption{Experiment with $J_d(x)$ on a subset of the SVHN2 dataset. For each normal class, training and testing sets containing 750 normal and 150 anomalous samples were used. To obtain the presented statistic, the subsets were sampled 5 times. The mean values of estimated $alpha$ weights are also presented and show that the weight of the jacobian term is suppressed during their computation.} 
 \label{tab:jacoceco_partial_experiment} 
\end{table}