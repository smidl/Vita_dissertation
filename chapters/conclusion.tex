\chapter{Conclusion} \label{sec:conclusion}
In this chapter, we first discuss the goals that were set in the introduction, and then the original contributions presented in this thesis.

\section{Evaluation of objectives}
The following goals were set in Chapter~\ref{sec:chapter_intro}. Overall, we believe that we have met them.

\begin{enumerate}
	\item Providing a compilation of the current state-of-the art for both classical (shallow) models and (deep) generative models for anomaly detection. This was done in Chapters~\ref{sec:chapter_shallow}~and~\ref{sec:chapter_survey}, where the models were described mostly from a theoretical point of view. 
	\item Conducting an extensive experimental comparison of selected methods under different operating conditions. While a small scale experiment on fusion physics data was presented at the end of Chapter~\ref{sec:alfven}, the results of a large empricial comparison were described in Chapter~\ref{sec:chapter_comparison}. These results show the direction of research that was further explored in Chapter~\ref{sec:chapter_sgvaegan}.
	\item Proposing a novel anomaly detector based on deep generative models. This model was introduced in Chapter~\ref{sec:chapter_sgvaegan}. Theoretical properties of its anomaly score as well as a detailed desciption of its training was provided.
\end{enumerate}


\section{Contributions}
There are several contributions contained in this work. It compiles several original peer-reviewed papers as well as some publications of the author that were not officialy published anywhere, but are publicly available.

In Chapter~\ref{sec:chapter_shallow}, the introduction to measures available for comparison of anomaly detectors is an excerpt from a paper~\cite{vskvara2023auc}, which is available at the ArXiv site. The goal of the paper was to determine whether the standard AUC is the most suitable measure for anomaly detector comparison, as it has some weaknesses. Besides the theoretical description present here, the paper also contains an extensive experimental comparison, which shows that although there are some niche usecases, where other measures are more suitable, the difference in the use of AUC or some other measure makes little difference. 

Furthermore, Chapters~\ref{sec:chapter_shallow} and~\ref{sec:chapter_survey} contain an extensive description of the current state-of-the-art shallow and deep anomaly detectors. Specifically Chapter~\ref{sec:chapter_survey} provides a thorough compilation of deep generative models and their use in anomaly detection. We believe it is useful both for novices and experienced researchers in the field of generative models and/or anomaly detection. We strived to distille our knowledge of this topic in a digestible yet comprehensive manner.

The Sec.~\ref{sec:alfven} compiles the publication~\cite{vskvara2020detection} on practical use of generative autoencoders in an anomaly detection problem in plasma fusion physics. Also, it provides some inspiration for the following chapters because it shows the feasibility of a two-stage approach, where a deep model is coupled with a shallow one operating on the latter model's low-dimensional representations of otherwise high-dimensional inputs.

Chapter~\ref{sec:chapter_comparison} contains the main results of the paper~\cite{vskvara2021comparison} which is a large-scale theoretical and experimental survey of deep generative models in anomaly detection. Its goals were to find a research direction in which generative anomaly detectors gain an advantage over other approaches, and it was succesful in doing so, as it inspired the model proposed in Chapter~\ref{sec:chapter_sgvaegan}. Also, the paper itself is a continuation of the original publication~\cite{vskvara2018generative}, which had the same goals, but a much lesser scope.

Finally, the last chapter contains theoretical and practical description of a novel SGVAEGAN model. This model comprises all the knowledge gained on generative anomaly detectors in the previous chapters and performs successfully on the problem of semantic anomaly detection on image data. At the time of the writing of this thesis, it is currently considered for publication.

Finally, one of the contributions of this work is that it contains links to all the related software developed during our research process. The list of publicly available repositories is in Appendix~\ref{sec:appendix_links}.

