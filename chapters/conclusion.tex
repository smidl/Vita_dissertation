\chapter{Conclusion} \label{sec:conclusion}
In this chapter, we summarize the individual chapters of the dissertation and their relationship to original contributions that were published as peer-reviewed papers or are publicly available in the form of preprints. Also, we try to evaluate the goals that were set in the introduction in Sec.~\ref{sec:objectives}. Finally, we offer an outlook on future work.

\section{Contributions}
\paragraph{Chapter~\ref{sec:chapter_intro}} This chapter sets the playing field for the rest of the text with definitions of basic terms and ideas that are important for anomaly detection. It also introduces basic types of anomalies and sets a list of objectives that we are going to evaluate.

\paragraph{Chapter~\ref{sec:chapter_shallow}} This chapter contains an introduction to measures available for comparison of anomaly detectors, which is an excerpt from the paper~\cite{vskvara2023auc}, where the goal was to determine whether the standard AUC is the most suitable measure for anomaly detector comparison, as it has some weaknesses. Besides the theoretical description present here, the paper also contains an extensive experimental comparison. Besides that, Chapter~\ref{sec:chapter_shallow} also contains an extensive description of the current state-of-the-art shallow anomaly detectors, which partly covers the first objective from Sec.~\ref{sec:objectives}.

\paragraph{Chapter~\ref{sec:chapter_survey}} The rest of the first objective is covered in this chapter with an extensive description of deep anomaly detectors. Special attention is given to detectors based on deep generative models. We believe it is useful both for novices and experienced researchers in the field of generative models and/or anomaly detection as we strived to distill our knowledge of this topic in a digestible yet comprehensive manner.

In Sec.~\ref{sec:alfven}, the practical use of generative autoencoders is presented in the context of anomaly detection problems in plasma fusion physics. It was published as a journal paper~\cite{vskvara2020detection}. It provides inspiration for the following chapters because it shows the feasibility of a two-stage approach, where a deep model is coupled with a shallow one operating on the latter model's low-dimensional representations of otherwise high-dimensional inputs.

\paragraph{Chapter~\ref{sec:chapter_comparison}} Here, the main results of the paper~\cite{vskvara2021comparison}, which is a large-scale theoretical and experimental survey of deep generative models in anomaly detection, are presented. Its goals were to find a research direction in which generative anomaly detectors gain an advantage over other approaches, and it was successful in doing so, as it inspired the model proposed in Chapter~\ref{sec:chapter_sgvaegan}. Also, this paper itself is a continuation of the original publication~\cite{vskvara2018generative}, which had the same goals, but a much lesser scope.

This chapter fully covers the second objective that was set in Sec.~\ref{sec:objectives} by providing insights into: i) the effect of building blocks of anomaly detection on their performance, and ii) the generally poor performance of the existing methods on semantic image anomalies, which is the focus of the following chapter.

\paragraph{Chapter~\ref{sec:chapter_sgvaegan}} This is the product of the previous chapters and the main contribution of the thesis, which also covers the last objective set in Sec.~\ref{sec:objectives}. In this chapter, we present a novel multi-factor anomaly detection scheme. The scheme is demonstrated on a novel SGVAEGAN model which performs successfully on the problem of semantic anomaly detection on image data. It also provides a novel manner of informed but unsupervised image disentanglement and most importantly a procedure for anomaly origin detection and explanation, which is something that is missing in the current state-of-art of image anomaly detection. At the time of the writing of this thesis, it is currently considered for publication.

Finally, one of the contributions of this work is that it contains links to all the related software developed during our research process. The list of publicly available repositories is in Appendix~\ref{sec:appendix_links}.

\section{Future work}
As the field of anomaly detection progresses, novel methods appear constantly. In the vein of Chapter~\ref{sec:chapter_comparison}, we should strive to add novel methods into a fair comparison that evaluates them from different perspectives. Although this is clearly not feasible to do continuously, an updated version of such a survey, created after a few years, would be helpful in assessing how the field of anomaly detection has evolved in the meantime. In fact, when publishing the original article~\cite{vskvara2021comparison}, we knew that there were novel methods that appeared during the publication process and which we would have liked to cover as well. Furthermore, the conclusion of Chapter~\ref{sec:chapter_comparison} covers some missing topics, such as anomaly detection with active learning or temporal anomaly detection, which is a research area that has been gaining a lot of momentum recently.

Possible technical improvements of the model proposed in Chapter~\ref{sec:chapter_sgvaegan}, such as the use of a more flexible latent prior, are mentioned at the end of that chapter. Although the SGVAEGAN model was specifically designed for anomaly detection on images where a prominent object is positioned on a distinct background, the general approach to anomaly detection via disentanglement based on Eq.~\eqref{eq:alphadeco} is possibly applicable to a different kind of data where there there is a possibility of anomalies coming from different independent sources. Finally, we believe that in order to improve the understanding of the proposed model behaviour, its capabilities would be best demonstrated in new experiments on some real--world application data.