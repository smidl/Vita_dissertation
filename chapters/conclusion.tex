\chapter{Conclusion} \label{sec:conclusion}
In this chapter, we first discuss the goals that were set in the introduction, and then the original contributions presented in this thesis.

\section{Evaluation of objectives}
The goals of the thesis set in Chapter~\ref{sec:chapter_intro} have been achieve as follows:

\begin{enumerate}
	\item Compilation of the current state-of-the art for both classical (shallow) models and (deep) generative models for anomaly detection was presented in Chapters~\ref{sec:chapter_shallow}~and~\ref{sec:chapter_survey}, where the theory of shallow, and deep models was reviews, respectively. The review also provided analysis of elementary approaches and building blocks of the existing methods.
	\item An extensive experimental comparison of existing methods under different operating conditions. While a small scale experiment on fusion physics data was presented at the end of Chapter~\ref{sec:alfven}, the results of a large empirical comparison were described in Chapter~\ref{sec:chapter_comparison}. These results provided insights into: i) the effect of building blocks of anomaly detection on their performance, and ii) generally poor performace of the existing method on semantic image anomalies. Semantic anomaly is a subset of all potential anomalies, hence we chose to investigate anomalies of various origins.  Chapter~\ref{sec:chapter_sgvaegan}.
	\item The novel anomaly detector based on deep generative models was introduced in Chapter~\ref{sec:chapter_sgvaegan}. It is a general method of multi-factor anomaly detection with experiments on image data using automatic image segmentation.  Theoretical properties of its anomaly score as well as a detailed desciption of its training was provided.
\end{enumerate}


\section{Contributions}
We now list individual contributions made in the thesis. Parts of the thesis have already been published in peer-reviewed papers as well as pre-prints that are publicly available.

In Chapter~\ref{sec:chapter_shallow}, the introduction to measures available for comparison of anomaly detectors is an excerpt from a paper~\cite{vskvara2023auc}, which is available at the ArXiv site. The goal of the paper was to determine whether the standard AUC is the most suitable measure for anomaly detector comparison, as it has some weaknesses. Besides the theoretical description present here, the paper also contains an extensive experimental comparison, which shows that although there are some niche usecases, where other measures are more suitable, the difference in the use of AUC or some other measure makes little difference. 

In Chapters~\ref{sec:chapter_shallow} and~\ref{sec:chapter_survey}, an extensive description of the current state-of-the-art shallow and deep anomaly detectors was presented. Specifically Chapter~\ref{sec:chapter_survey} provides a thorough compilation of deep generative models and their use in anomaly detection. We believe it is useful both for novices and experienced researchers in the field of generative models and/or anomaly detection. We strived to distille our knowledge of this topic in a digestible yet comprehensive manner. \vs{analysis -- building blocks?}

In Sec.~\ref{sec:alfven}, practical use of generative autoencoders is presented in the context of anomaly detection problem in plasma fusion physics. It was published as a journal paper~\cite{vskvara2020detection}. It provides some inspiration for the following chapters because it shows the feasibility of a two-stage approach, where a deep model is coupled with a shallow one operating on the latter model's low-dimensional representations of otherwise high-dimensional inputs.

In Chapter~\ref{sec:chapter_comparison}, the main results of the paper~\cite{vskvara2021comparison} which is a large-scale theoretical and experimental survey of deep generative models in anomaly detection is presented. Its goals were to find a research direction in which generative anomaly detectors gain an advantage over other approaches, and it was succesful in doing so, as it inspired the model proposed in Chapter~\ref{sec:chapter_sgvaegan}. Also, the paper itself is a continuation of the original publication~\cite{vskvara2018generative}, which had the same goals, but a much lesser scope.

In Chapter~\ref{sec:chapter_sgvaegan}, the main contribution of the thesis in the form of a novel multi-factor anomaly detection scheme is presented. The scheme is demonstrated on a novel SGVAEGAN model which comprises all the knowledge gained on generative anomaly detectors in the previous chapters and performs successfully on the problem of semantic anomaly detection on image data. It also provides a novel manner of informed but unsupervised image disentanglement and most importantly a procedure for anomaly origin detection and explanation, which is something that is missing in the current state-of-art of image anomaly detection. At the time of the writing of this thesis, it is currently considered for publication.

Finally, one of the contributions of this work is that it contains links to all the related software developed during our research process. The list of publicly available repositories is in Appendix~\ref{sec:appendix_links}.

