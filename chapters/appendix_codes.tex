\chapter{Additional resources} \label{sec:appendix_links}

Most of the datasents from Appendix~\ref{sec:appendix_datasets} are publicly available. The Wildlife MNIST and COCOPlaces datasets were created and published by the author of this thesis and are available at \url{https://zenodo.org/record/7602025} and \url{https://zenodo.org/record/7612053}. The data that were produced during the extensive experimental comparison are not publicly available, since they exceed 10TB in size, but the model performance metrics from experiments in Chapters~\ref{sec:chapter_comparison} and~\ref{sec:chapter_sgvaegan} are available upon request from the authors.

Most of the models which wre experimented with experiments were implemented either mostly in the Julia~\cite{Julia-2017} language, with the exception of the SGVAEGAN model, which was implemented in PyTorch~\cite{NEURIPS2019_9015}. Now we list the publicly available repositories used in the creation of this thesis.

\begin{enumerate}
    \item Repository \url{https://github.com/vitskvara/UCI.jl} provides an easy access to anomaly benchmark datasets that were created from classification problems in the UCI dataset database. This was used for the experiments with tabular data in Chapter~\ref{sec:chapter_comparison}.
    \item Repository \url{https://github.com/vitskvara/AlfvenDetectors.jl} is a collection of model code and utilities for experiments on Alfvén detection that was described in Sec.~\ref{sec:alfven}.
    \item Repository \url{https://github.com/aicenter/GenerativeModels.jl} contains a very general interface for training and evaluation of basic generative models.
    \item Repository \url{github.com/vitskvara/GenerativeAD.jl} compiles the experimental framework used for the large scale experimental comparisons in Chapters~\ref{sec:chapter_comparison} and~\ref{sec:chapter_sgvaegan}.
    \item Finally, the repository \url{github.com/vitskvara/sgad} contains the PyTorch implementation of the SGVAEGAN model.
\end{enumerate}

